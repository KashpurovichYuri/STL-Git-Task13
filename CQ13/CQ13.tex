\documentclass[a4paper,12pt]{article}	% тип документа

\usepackage[a4paper,top=1.3cm,bottom=2cm,left=1.5cm,right=1.5cm,marginparwidth=0.75cm]{geometry} % field settings

\usepackage[T2A]{fontenc}		% кодировка
\usepackage[utf8]{inputenc}		% кодировка исходного текста
\usepackage[english,russian]{babel}	% локализация и переносы
\usepackage{indentfirst}

%Piece of code
\usepackage{listings}
\usepackage{xcolor}
\lstset
{
    language=C++,
    backgroundcolor=\color{black!4}, % set backgroundcolor
    basicstyle=\footnotesize,% basic font setting
}

%Drawings
\usepackage{graphicx}

\usepackage{wrapfig}

\usepackage{multirow}

\usepackage{float}

\usepackage{wasysym}

\usepackage[T1]{fontenc}
\usepackage{titlesec}

\setlength{\parindent}{3ex}

% Literature
\addto\captionsrussian{\def\refname{Литература.}}

%Header
\title{
	\center{\textbf{Контрольные вопросы. Задание 13.}}
	}


\begin{document}	% the beginning of the document

\maketitle

\section{Какие концепции лежат в основе стандартной библиотеки?}
	
	Концепции, лежащие в основе стандартной библиотеки:
	
	\begin{itemize}
	
		\item переносимость;
		
		\item компактность/эффективность;
		
		\item связывание.
	
	\end{itemize}
	
	При этом реализации стандартной библиотеки могут отличаться, но они всегда обеспечивают одинаковое поведение.
		
\newpage

\section{Зачем в проектах используются системы контроля версий?}

	Системы контроля версий позволяют удобным, логичным и контролируемым образом (можно контролировать историю разработки, права доступа, различные версии) обмениваться рабочими материалами (программами, файлами) при реализации того или иного проекта, поэтому они и используются, причём как в одиночных проектах, так и в более крупных.
	
\newpage

\section{Из каких основных действий состоит взаимодействие с git?}
	
	Основные действия с git:
	
	\begin{itemize}
	
		\item clone -- создание копии репозитория.
		
		\item commit -- сохранение последних изменений файла в репозитории.
		
		\item push -- загрузка содержимого локального репозитория в удалённый.
		
		\item pull -- загрузка содержимого удалённого репозитория в локальный. При этом любые внесённые коммиты сливаются в ветку, в которой сейчас работает разработчик.
	
	\end{itemize}
	
	Последние две команды обеспечивают синхронизацию содержимого локального и удалённого репозиториев. Также можно выделить команды fetch, merge и rebase. Первая из них собирает все коммиты из целевой ветки, которых нет в текущей ветке, и сохраняет их в локальном репозитории, не объединяя их в текущей ветку. Последние отвечают за слияние ветвей разработки.

\newpage

\section{Когда следует создавать отдельные ветки для разработки?}

	Отдельные ветки для разработки следует создавать для параллельной разработки нескольких версий программы (контроль версий). Например, одна версия может быть стабильной для работы с клиентами, в ней будут лишь исправляться ошибки. При этом параллельные ветви могут быть использованы для улучшения каких-либо характеристик программы, для грядущих изменений (после их отладки).
	
\newpage

\section{Какие основные элементы содержатся в библиотеке chrono?}
	
	Основные элементы библиотеки chrono:
	
	\begin{itemize}
	
		\item clock (таймер) -- тип, определяющий эпоху и продолжительность такта.
		
		\item duration (интервал) -- тип, определяющий промежуток времени, состоящий из некоторого количества тактов заданной продолжительности.
		
		\item timepoint (момент времени) -- тип, определяющий продолжительность времени, прошедшего с начала отсчёта (эпохи). Иными словами, это комбинация эпохи и соответствующего интервала времени (отсчитываемого от начала отсчёта).
	
	\end{itemize}
	
\newpage


\addcontentsline{toc}{section}{Литература}
 
	\begin{thebibliography}{}
	
		\bibitem{litlink1} Конспект семинара. Макаров И.С.
		\bibitem{litlink2} https://en.cppreference.com/w/cpp/chrono.
		\bibitem{litlink3} https://tproger.ru/explain/git-pull-and-git-fetch-whats-the-difference/		
		
	\end{thebibliography}


\end{document} % end of the document
